\documentclass[18pt,a4paper,oneside,UTF8]{ctexart}
\author{Leon}
\title{线性回归}
\begin{document}
\maketitle
\section{机器学习的概念}
\subsection{有监督学习}
对训练集来说X对应着是确定的f(x),然后通过构建模型,进行超参的学习
\subsection{无监督学习}
大部分无监督学习都是没有确定的f(x)的,通过一些规则,让机器自己去判断,比如knn算法,用距离来做聚类。
\subsection{泛化能力}
在机器学习方法中,泛化能力通俗来讲就是指学习到的模型对未知数据的预测能力。在实际情况中,我们通常通过测试误差来评价学习方法的泛化能力。
\subsection{过拟合}
\subsubsection{概念}
先谈谈过拟合,所谓过拟合,指的是模型在训练集上表现的很好,但是在交叉验证集合测试集上表现一般,也就是说模型对未知样本的预测表现一般,泛化(generalization)能力较差。
\subsubsection{解决办法}
一般的方法有early stopping、数据集扩增(Data augmentation)、正则化(Regularization)、Dropout等。
在机器学习算法中,我们常常将原始数据集分为三部分:training data、validation data,testing data。这个validation data是什么?它其实就是用来避免过拟合的,在训练过程中,我们通常用它来确定一些超参数(比如根据validation data上的accuracy来确定early stopping的epoch大小、根据validation data确定learning rate等等)。那为啥不直接在testing data上做这些呢?因为如果在testing data做这些,那么随着训练的进行,我们的网络实际上就是在一点一点地overfitting我们的testing data,导致最后得到的testing accuracy没有任何参考意义。
\paragraph{Early stopping:}
Early stopping便是一种迭代次数截断的方法来防止过拟合的方法,即在模型对训练数据集迭代收敛之前停止迭代来防止过拟合。对模型进行训练的过程即是对模型的参数进行学习更新的过程,这个参数学习的过程往往会用到一些迭代方法,如梯度下降(Gradient descent)学习算法。这样可以有效阻止过拟合的发生,因为过拟合本质上就是对自身特点过度地学习。
\paragraph{正则化:}
指的是在目标函数后面添加一个正则化项,一般有L1正则化与L2正则化。L1正则是基于L1范数,即在目标函数后面加上参数的L1范数和项,即参数绝对值和与参数的积项
\[
    C=C_0+\frac {\lambda }{n} \sum_w |w|
\]
L2正则是基于L2范数,即在目标函数后面加上参数的L2范数和项,即参数的平方和与参数的积项:
\[
    C=C_0+\frac {\lambda}{2n} \sum_w w^2    
\]
\subsection{交叉验证(cross-validation)}
交叉验证,是重复的使用数据,把得到的样本数据进行切分,组合为不同的训练集和测试集,用训练集来训练模型,用测试集来评估模型预测的好坏。在此基础上可以得到多组不同的训练集和测试集,某次训练集中的某样本在下次可能成为测试集中的样本,即所谓“交叉”。有简单交叉验证、S折交叉验证、留一交叉验证。
\subsection{线性回归的原理}
1:函数模型(Model):
\[
    h_w(x^i)=\omega_0+\omega_1 x_1+\omega_2 x_2+...+\omega_n x_n=\sum \omega^T x_i=W^T X 
\]
\begin{equation} 
X={ \left[ \begin{array}{ccc} 1\\x_1\\...\\x_n \end{array} \right ]}, W={ \left[ \begin{array}{ccc} \omega_0\\
    \omega_2
    \\...
    \\\omega_n 
\end{array} \right ]} 
\end{equation}
假设有训练数据
\[
    D={(X_1,Y_1),(X_2,Y_2),...,(X_n,Y_n)}    
\]
那么方便我们写成矩阵的形式
\[
    X={\left [\begin{array}{ccc}
        1,x^1_n,x^1_2,...,x^1_n\\
        1,x^2_1,x^2_2,...,x^2_n\\
        .......\\
        1,x^n_1,x^n_2,...x^n_n
    \end{array} \right]}  
    ,XW=h_\omega(x^i)    
\]
2.损失代价函数:
\[
    \emph{J}(W)=\frac {1}{2M}\sum_{i=0}^M(h_\omega(x^i)-y^i)^2=\frac {1}{2M}(XW-y)^T(XW-Y)
\]
3.算法(algorithm):
求解使得损失函数最小。
\subsection{优化方法}
\subsubsection{梯度下降法}
梯度下降沿损失函数的导数方向下降,下降的步幅自己设置。
\subsubsection{牛顿法}
二阶下降,比梯度下降法更快,而且是求全局最优解,不是局部最优
\subsubsection{拟牛顿法}
没看懂,但知道适合非线性
\subsection{sklearn参数}
Ordinary least squares Linear Regression.
\subsubsection{fit\_intercept:boolean, optional, default True}
\paragraph{}whether to calculate the intercept for this model. If set
to False, no intercept will be used in calculations
(e.g. data is expected to be already centered).
\subsubsection{normalize : boolean, optional, default False}
\paragraph{}This parameter is ignored when ``fit\_intercept`` is set to False.
If True, the regressors X will be normalized before regression by
subtracting the mean and dividing by the l2-norm.
If you wish to standardize, please use
:class:`sklearn.preprocessing.StandardScaler` before calling ``fit`` on
an estimator with ``normalize=False``.
\subsubsection{copy\_X : boolean, optional, default True}
\paragraph{}If True, X will be copied; else, it may be overwritten.
\subsubsection{n\_jobs : int or None, optional (default=None)}
\paragraph{}The number of jobs to use for the computation. This will only provide
speedup for n\_targets > 1 and sufficient large problems.``None`` means 1.
\end{document}
