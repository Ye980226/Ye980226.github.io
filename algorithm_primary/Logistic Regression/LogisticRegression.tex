\documentclass[18pt,a4paper,oneside,UTF8]{ctexart}
\author{Leon}
\title{逻辑回归}
\begin{document}
\maketitle
\section{机器学习的概念}
\subsection{有监督学习}
对训练集来说X对应着是确定的f(x),然后通过构建模型,进行超参的学习
\subsection{无监督学习}
大部分无监督学习都是没有确定的f(x)的,通过一些规则,让机器自己去判断,比如knn算法,用距离来做聚类。
\subsection{泛化能力}
在机器学习方法中,泛化能力通俗来讲就是指学习到的模型对未知数据的预测能力。在实际情况中,我们通常通过测试误差来评价学习方法的泛化能力。
\subsection{过拟合}
\subsubsection{概念}
先谈谈过拟合,所谓过拟合,指的是模型在训练集上表现的很好,但是在交叉验证集合测试集上表现一般,也就是说模型对未知样本的预测表现一般,泛化(generalization)能力较差。
\subsubsection{解决办法}
一般的方法有early stopping、数据集扩增(Data augmentation)、正则化(Regularization)、Dropout等。
在机器学习算法中,我们常常将原始数据集分为三部分:training data、validation data,testing data。这个validation data是什么?它其实就是用来避免过拟合的,在训练过程中,我们通常用它来确定一些超参数(比如根据validation data上的accuracy来确定early stopping的epoch大小、根据validation data确定learning rate等等)。那为啥不直接在testing data上做这些呢?因为如果在testing data做这些,那么随着训练的进行,我们的网络实际上就是在一点一点地overfitting我们的testing data,导致最后得到的testing accuracy没有任何参考意义。
\paragraph{Early stopping:}
Early stopping便是一种迭代次数截断的方法来防止过拟合的方法,即在模型对训练数据集迭代收敛之前停止迭代来防止过拟合。对模型进行训练的过程即是对模型的参数进行学习更新的过程,这个参数学习的过程往往会用到一些迭代方法,如梯度下降(Gradient descent)学习算法。这样可以有效阻止过拟合的发生,因为过拟合本质上就是对自身特点过度地学习。
\paragraph{正则化:}
指的是在目标函数后面添加一个正则化项,一般有L1正则化与L2正则化。L1正则是基于L1范数,即在目标函数后面加上参数的L1范数和项,即参数绝对值和与参数的积项
\[
    C=C_0+\frac {\lambda }{n} \sum_w |w|
\]
L2正则是基于L2范数,即在目标函数后面加上参数的L2范数和项,即参数的平方和与参数的积项:
\[
    C=C_0+\frac {\lambda}{2n} \sum_w w^2    
\]
\subsection{交叉验证(cross-validation)}
交叉验证,是重复的使用数据,把得到的样本数据进行切分,组合为不同的训练集和测试集,用训练集来训练模型,用测试集来评估模型预测的好坏。在此基础上可以得到多组不同的训练集和测试集,某次训练集中的某样本在下次可能成为测试集中的样本,即所谓“交叉”。有简单交叉验证、S折交叉验证、留一交叉验证。
\subsection{线性回归的原理}
1:函数模型(Model):
\[
    h_w(x^i)=\omega_0+\omega_1 x_1+\omega_2 x_2+...+\omega_n x_n=\sum \omega^T x_i=W^T X 
\]
\begin{equation} 
X={ \left[ \begin{array}{ccc} 1\\x_1\\...\\x_n \end{array} \right ]}, W={ \left[ \begin{array}{ccc} \omega_0\\
    \omega_2
    \\...
    \\\omega_n 
\end{array} \right ]} 
\end{equation}
假设有训练数据
\[
    D={(X_1,Y_1),(X_2,Y_2),...,(X_n,Y_n)}    
\]
那么方便我们写成矩阵的形式
\[
    X={\left [\begin{array}{ccc}
        1,x^1_n,x^1_2,...,x^1_n\\
        1,x^2_1,x^2_2,...,x^2_n\\
        .......\\
        1,x^n_1,x^n_2,...x^n_n
    \end{array} \right]}  
    ,XW=h_\omega(x^i)    
\]
2.损失代价函数:
\[
    \emph{J}(W)=\frac {1}{2M}\sum_{i=0}^M(h_\omega(x^i)-y^i)^2=\frac {1}{2M}(XW-y)^T(XW-Y)
\]
3.算法(algorithm):
求解使得损失函数最小。
\subsection{优化方法}
\subsubsection{梯度下降法}
梯度下降沿损失函数的导数方向下降,下降的步幅自己设置。
\subsubsection{牛顿法}
二阶下降,比梯度下降法更快,而且是求全局最优解,不是局部最优
\subsubsection{拟牛顿法}
没看懂,但知道适合非线性
\subsection{sklearn参数}
Ordinary least squares Linear Regression.
\subsubsection{fit\_intercept:boolean, optional, default True}
\paragraph{}whether to calculate the intercept for this model. If set
to False, no intercept will be used in calculations
(e.g. data is expected to be already centered).
\subsubsection{normalize : boolean, optional, default False}
\paragraph{}This parameter is ignored when ``fit\_intercept`` is set to False.
If True, the regressors X will be normalized before regression by
subtracting the mean and dividing by the l2-norm.
If you wish to standardize, please use
:class:`sklearn.preprocessing.StandardScaler` before calling ``fit`` on
an estimator with ``normalize=False``.
\subsubsection{copy\_X : boolean, optional, default True}
\paragraph{}If True, X will be copied; else, it may be overwritten.
\subsubsection{n\_jobs : int or None, optional (default=None)}
\paragraph{}The number of jobs to use for the computation. This will only provide
speedup for n\_targets > 1 and sufficient large problems.``None`` means 1.
\section{逻辑回归}
\subsection{逻辑回归和线性回归的联系与区别}
\subsubsection{联系}
线性回归和逻辑回归都是广义线性模型,具体的说,都是从指数分布族导出的线性模型,线性回归假设Y|X服从高斯分布,逻辑回归假设Y|X服从伯努利分布,这两种分布都是属于指数分布族,我们可以通过指数分布族求解广义线性模型(GLM)的一般形式,导出这两种模型。
\subsubsection{区别}
区别是线性回归是用来做预测任务,逻辑回归是用来做分类任务,把每一个点映射到0-1之间,都是用最大似然概率去求解参数值。
\subsection{逻辑回归的原理}
线性回归的模型是求出输出特征向量Y和输入样本矩阵X之间的线性关系系数θ,满足$Y=X\theta^T$。我们的Y是连续的,所以是回归模型。如果Y是离散的话,可以想到的办法是,我们对于这个Y再做一次函数转换,变为g(Y)。如果我们令g(Y)的值在某个实数区间的时候是类别A,在另一个实数区间的时候是类别B,以此类推,就得到了一个分类模型。如果结果的类别只有两种,那么就是一个二元分类模型了。逻辑回归的出发点就是从这来的。下面我们开始引入二元逻辑回归。也就是输入X,输出的Y,只有1,0两种情况,而线性回归的X*系数矩阵得到的值如果直接通过比较得出Y为1或0,是可以做,但是得到的结果无法进行再次学习,或者说很麻烦,优化方法跟导数有关,所以如果要优化,就得打造一个完美的连续函数,比如sigmoid,方便求导,而且两个极限值是0和1,但是有一个不好的地方,有可能求解到局部最优。
\subsection{逻辑回归损失函数推导及优化}
对线性回归的结果做一个函数g上的转换,可以变为逻辑回归,一般取为sigmoid函数,形式如下:
\[
    g(z)=\frac {1}{1+e^{-z}}
\]
它有一个非常好的导数性质:
\[
    g'(z)=g(z)(1-g(z))    
\]
这个通过函数对g(z)求导很容易得到
如果我们令g(z)中的z为:$ z=x\theta$,这样就得到了二元逻辑回归模型的一般形式:
\[
    h_{\theta}(x)=\frac {1}{1+e^{-x\theta}}
\]
其中x为样本输入,$h_\theta(x)$为模型输出,可以理解为某一分类的概率大小。而$\theta$为分类模型的要求出的模型参数。对于模型输出$ h_\theta(x) $,我们让它和我们的二元样本输出y(假设0和1)有这样的对应关系,如果$h_\theta(x)$>0.5,即$x\theta$>0,则y为1,反之亦然,y=0.5是临界情况,此时无法确定分类,$h_\theta(x)$值越小,而分类为0的概率越高,反之,值越大的话分类为1的概率越高。如果靠近临界点,则分类准确率会下降。
\paragraph{}
此处将模型写成矩阵模式:
$h_\theta(X)=\frac {1}{1+e^{-X\theta}}$
\paragraph{}
假设我们的样本输出是0或者1两类,那么我们有:
\[
    P(y=1|x,\theta)=h_\theta(x)    
\]
\[
    P(y=0|x,\theta)=1-h_\theta(x)    
\]
把这两个式子写成一个式子,就是:
\[
    P(y|x,\theta)=h_\theta(x)^y(1-h_\theta(x))^{1-y}    
\]
其中y的取值只能是0或者1。
用矩阵法表示,即为
\[
    P(Y|X,\theta)=h_\theta(X)^Y(E-h_\theta(X))^{1-Y},其中E为单位向量。    
\]
得到了y的概率分布函数表达式,我们就可以用似然函数最大化来求解我们需要的模型系数θ。
为了方便求解,这里用对数似然函数最大化,对数似然函数取反即为我们的损失函数J(θ)。其中:
\[
    L(\theta)=\prod_{i=1}^m(h_\theta(x^{(i)}))^{y(i)}(1-h_\theta(x^{(i)}))^{1-y^{(i)}}  
\]
其中m为样本的个数
对似然函数对数化取反的表达式,即损失函数表达式为:
\[
    J(\theta)=-lnL(\theta)=-\sum_{i=1}^m(y^{(i)}log(h_\theta(x^{(i)}))+(1-y^{(i)})log(1-h_\theta(x^{(i)})))    
\]
损失函数用矩阵法表达更加简洁:\newline
J(θ)=−Y⊙loghθ(X)−(E−Y)⊙log(E−hθ(X))\newline
其中E为全1向量,⊙为哈达马乘积(对应位置相乘)。\newline
对于J(θ)=−Y⊙loghθ(X)−(E−Y)⊙log(E−hθ(X)),我们用J(θ)对θ\newline
向量求导可得:\newline

$\frac {\partial{J(\theta)}}{\partial \theta}=−Y\bigodot X^T \frac {1}{h_\theta(X)}\bigodot h_\theta(X)\bigodot(E−h_\theta(X))+(E−Y)\bigodot X^T \frac{1}{1−h_\theta(X)}\bigodot h_\theta(X)\bigodot(E−h_\theta(X))$\newline
简化得到\newline
$\frac {\partial}{\partial \theta} J(\theta)=X^T(h_\theta(X)-Y)$\newline
从而在梯度下降法中每一步向量$\theta$的迭代公式如下:\newline
$\theta=\theta-\alpha X^T(h_\theta(X)-Y)$\newline
其中,$\alpha$为梯度下降法的步长。
\subsection{正则化与模型评估指标}
在最小化残差平方和的基础上加上L1范数或者L2范数的惩罚项,如果L2正则正则化就是-岭回归 Ridge Regression,L1正则化是lasso回归。
回归模型评估指标有\newline
1.解释方差\newline
$Explained_variance(y,y_hat)=1-Var(y-y_hat)/Var(y)$
2.绝对平均误差\newline
3.均方误差\newline
4.决定系数($R^2$ score)\newline
5.AIC\newline
6.BIC\newline
\subsection{逻辑回归的优缺点} 

优点:\newline\newline
1)预测结果是界于0和1之间的概率;\newline\newline
2)可以适用于连续性和类别性自变量;\newline\newline
3)容易使用和解释;\newline\newline

缺点:\newline\newline
1)对模型中自变量多重共线性较为敏感,例如两个高度相关自变量同时放入模型,可能导致较弱的一个自变量回归符号不符合预期,符号被扭转。​需要利用因子分析或者变量聚类分析等手段来选择代表性的自变量,以减少候选变量之间的相关性;\newline\newline
2)预测结果呈“S”型,因此从log(odds)向概率转化的过程是非线性的,在两端随着​log(odds)值的变化,概率变化很小,边际值太小,slope太小,而中间概率的变化很大,很敏感。 导致很多区间的变量变化对目标概率的影响没有区分度,无法确定阀值。\newline\newline
\subsection{样本不均衡问题的解决办法}
分类时,由于训练集合中各样本数量不均衡,导致模型训偏在测试集合上的泛化性不好。解决样本不均衡的方法主要包括两类:(1)数据层面,修改各类别的分布;(2)分类器层面,修改训练算法或目标函数进行改进。还有方法是将上述两类进行融合。
\subsubsection{数据层面}

过采样

    1.基础版本的过采样:随机过采样训练样本中数量比较少的数据;缺点,容易过拟合;

    2.改进版本的过采样:SMOTE,通过插值的方式加入近邻的数据点;

    3.基于聚类的过采样:先对数据进行聚类,然后对聚类后的数据分别进行过采样。这种方法能够降低类间和类内的不平衡。

    4.神经网络中的过采样:SGD训练时,保证每个batch内部样本均衡。

    欠采样

与过采样方法相对立的是欠采样方法,主要是移除数据量较多类别中的部分数据。这个方法的问题在于,丢失数据带来的信息缺失。为克服这一缺点,可以丢掉一些类别边界部分的数据。
\subsubsection{分类器层面}
过采样,欠采样,都存在相应的问题。

过采样:可能会存在过拟合问题。(可以使用SMOTE算法,增加随机的噪声的方式来改善这个问题)

欠采样:可能会存在信息减少的问题。因为只是利用了一部分数据,所以模型只是学习到了一部分模型。

有以下两种方法可以解决欠采样所带来的问题。

方法一:模型融合 (bagging的思想 )

思路:从丰富类样本中随机的选取(有放回的选取)和稀有类等量样本的数据。和稀有类样本组合成新的训练集。这样我们就产生了多个训练集,并且是互相独立的,然后训练得到多个分类器。

若是分类问题,就把多个分类器投票的结果(少数服从多数)作为分类结果。

若是回归问题,就将均值作为最后结果。

方法二:增量模型 (boosting的思想)

思路:使用全部的样本作为训练集,得到分类器L1

从L1正确分类的样本中和错误分类的样本中各抽取50%的数据,即循环的一边采样一个。此时训练样本是平衡的。训练得到的分类器作为L2.

从L1和L2分类结果中,选取结果不一致的样本作为训练集得到分类器L3.

最后投票L1,L2,L3结果得到最后的分类结果。

\end{document}
